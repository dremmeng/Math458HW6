% LaTeX Article Template - customizing page format
%
% LaTeX document uses 10-point fonts by default.  To use
% 11-point or 12-point fonts, use \documentclass[11pt]{article}
% or \documentclass[12pt]{article}.
\documentclass{article}

% Set left margin - The default is 1 inch, so the following 
% command sets a 1.25-inch left margin.
\setlength{\oddsidemargin}{0.25in}

% Set width of the text - What is left will be the right margin.
% In this case, right margin is 8.5in - 1.25in - 6in = 1.25in.
\setlength{\textwidth}{6in}

% Set top margin - The default is 1 inch, so the following 
% command sets a 0.75-inch top margin.
\setlength{\topmargin}{-0.25in}

% Set height of the text - What is left will be the bottom margin.
% In this case, bottom margin is 11in - 0.75in - 9.5in = 0.75in
\setlength{\textheight}{8in}
\usepackage{fancyhdr, lastpage}
\usepackage{tikz}
\usepackage{amsmath,amssymb,amsthm}
\usetikzlibrary{calc}
\usepackage{enumitem}
\usepackage{soul}
\usetikzlibrary{positioning}
\graphicspath{ {./} }
\setlength{\parskip}{5pt} 
\pagestyle{fancyplain}

% Universes
\newcommand{\NN}{\mathbb{N}}
\newcommand{\ZZ}{\mathbb{Z}}
\newcommand{\QQ}{\mathbb{Q}}
\newcommand{\RR}{\mathbb{R}}
\newcommand{\CC}{\mathbb{C}}

% Groups commands
\newcommand{\inv}{^{-1}}
\newcommand{\lcm}{\mathrm{lcm}}
\newcommand{\lr}[1]{\langle #1 \rangle}
\newcommand{\Inn}{\mathrm{Inn}}
\newcommand{\iso}{\cong}
% Set the beginning of a LaTeX document
\begin{document}

\lhead{Drew Remmenga MATH 458}
\rhead{HW \#6}
%\lhead{Independent Study}
%\rhead{R Lab}


\begin{enumerate}

\item Find a subgroup of $\ZZ_{20} \oplus U(16)$ that is isomorphic to $\ZZ_4\oplus\ZZ_5$. Provide the isomorphism (but you do not need to prove your map works). Let $H = \{0,4,8,12,16\}$ which is a subgroup of $\ZZ_{20}$ of order 5. Let $K = \{1,5,9,13\}$ which is a subgroup of U(16) of order 4. Now $H \oplus K$ is a subgroup of $G= \ZZ_{20} \oplus U(16)$. Now define isomophism $\phi : \ZZ_{4} \oplus \ZZ_{5} \rightarrow K \oplus H$ as $\phi (x,y) = (5^{x},4y)$. Then $\phi$ is an ismorphism.

    
\item Consider the group $\ZZ_{90}\oplus\ZZ_{36}$.
    \begin{enumerate}
        \item Without computing all of them, determine how many elements of order 15 are there in $\ZZ_{90}\oplus\ZZ_{36}$? 32 elements of order 15. 
        
        
        \item Determine the number of cyclic groups of order 15 in $\ZZ_{90}\oplus\ZZ_{36}$. {\color{red}\st{Provide a generator for each of the subgroups.}} 4 cyclic groups.

    \end{enumerate}
    
    
\item In this problem, we show why the operation we defined on cosets only makes sense when the subgroup is normal. 
    \begin{enumerate} 
        \item Let $H$ be a subgroup of a group $G$ with the property that for all $a,b\in G$, $aHbH=abH$. Prove that $H$ must be normal. Consider any element $g \in G$. Then $gH$ and $g^{-1} H$ are two left cosets of H in G. This $gHg^{-1}H$ is also a left coset of $H$ in $G$. Also  $gHg^{-1}H = g g^{-1} H = e H$. Now $eH = geg^{-1} H \in gHg^{-1}H \implies e \in gHg^{-1}H$. Also $e \in H$ as $H$ is a subgroup of $G$. So we get $H$ and $gHg^{-1}H$ are two left cosets having one common elemeent $e$. We know from the property of equivalence classes that two left cosets are either equal or have no elements in common. Therefore $H = gHg^{-1}H$. Now for all $h_1, h_2 \in H$ and $g \in G$, $gh_1g^{-1}h_2 = gHg^{-1}H \implies gh_1g^{-1}h_2 \in H \implies gh_1g^{-1}h_2h^{-1}_{2} = Hh_{2}^{-1} \implies gh_1g^{-1} \in H$ for all $h_1 \in H$ and all $g \in G$ implies $H$ is normal in $G$. 
        
    
        \item Give an example of a group $G$ and subgroup $K$ such that $aKbK\ne abK$ for some $a,b\in G$. $G = \mathbb{R}^{*}$ $K= 2^{x}$ for all integers.  

        
    \end{enumerate} 
    
    
\item Let $H$ be a normal subgroup of a finite group $G$ and let $x\in G$. If $\gcd(|x|, |G/H|)=1$, show that $x\in H$. Let $|x| =a$ and $|G/H|=b$. We are given that $\gcd(|x|, |G/H|)=1$ so we know there exists $m,n \in \mathbb{Z}$ such that $ma + nb =1$. Now consider the coset $xH$.  $(xH)^{a} = eH = H \implies x^{a} \in H$. Likewise we have $(xH)^{b} = H \implies x^{b} \in H$. Now we can write $x = x^{1} = x^{ma+nb} = x^{am} x^{bn} \in H$. Since $x^{a} \in H$ $(x^{a})^{m} \in H$. Similarly $(x^{b})^{n} \in H$ so $x \in H$. 
    
    
\item The following theorem and proof is presented in the textbook (Theorem~9.7 in the 9th edition).
    
    \bigskip
    {\sffamily
    \textit{Theorem. }
    Let $G$ be group of order $p^2$ where $p$ is prime. Then $G$ is isomorphic to $\ZZ_{p^2}$ or $\ZZ_p\oplus \ZZ_p$.

    \textit{Proof.}
    Let $G$ be a group of order $p^2$, where $p$ is a prime. If $G$ has an element of order $p^2$, then $G$ is isomorphic to $\ZZ_{p^2}$.
    \textbf{(1) So, by Corollary 2 of Lagrange’s Theorem, we may assume that every nonidentity element of $G$ has order $p$. }
    First we show that for any element $a$, the subgroup $\langle a\rangle$ is normal in $G$. 
    \textbf{(2) If this is not the case, then there is an element $b$ in $G$ such that $bab\inv$ is not in $\langle a\rangle$.}
    \textbf{(3) Then $\langle a\rangle$ and $\langle bab\inv\rangle$ are distinct subgroups of order $p$.}
    Since $\langle a\rangle\cap\langle bab\inv\rangle$ is a subgroup of both $\langle a\rangle$ and $\langle bab\inv\rangle$, we have that $\langle a\rangle\cap\langle bab\inv\rangle=\{e\}$. 
    \textbf{(4) From this it follows that the distinct left cosets of $\langle bab\inv\rangle$ are $\langle bab\inv\rangle$, $a\langle bab\inv\rangle$, $a^2\langle bab\inv\rangle$ , \dots, $a^{p-1}\langle bab\inv\rangle$.} 
    Since $b\inv$ must lie in one of these cosets, we may write $b\inv$ in the form $b\inv=a^i(bab\inv)^j=a^iba^jb\inv$ for some $i$ and $j$. 
    Canceling the $b\inv$ terms, we obtain $e=a^iba^j$ and therefore $b=a^{-i-j}\in\langle a\rangle$. 
    \textbf{(5) This contradiction verifies our assertion that every subgroup of the form $\langle a\rangle$ is normal in $G$. }
    To complete the proof, let $x$ be any nonidentity element in $G$ and $y$ be any element of $G$ not in $\langle x\rangle$.
    \textbf{(6) Then, by comparing orders and using Theorem 9.6, we see that $G=\langle x\rangle\times\langle y\rangle\iso \ZZ_p\oplus\ZZ_p$.}
    }
    \bigskip
    
    The author left out a lot of the details in the proof (as usual).  Most sentences could use more explanation, but the 6 sentences in bold in particular require more justification. Replace those sentences with the missing details to write a more detailed proof (and add some paragraph spacing to make it more readable).
    In a finite group the order of each element divides the order of the group. Assume the subgroup $\langle a \rangle$ is normal in $G$. The subgroup is normal iff $gag^{-1} = \langle a \rangle$. For all $g \in G$. That won't turn out to be correct because we willl have an elemenet $b$ in $\langle a \rangle$ such that it has order p. So $bab^{-1} \not \in \langle a \rangle$ as it will not satify the condition of a normal subgroup. Now as $bab^{-1} \not \in \langle a \rangle$ we have two subgroups of order p namely $\langle bab^{-1}\rangle$ and $\langle a \rangle$ because the order couldn't be one and the only other choice is p. Since the intersection of two subgroups is always a subgroup we are left with these two subgroups having an order of one namely it is the identity. $b^{-1}$ must lie in one of these two cosets. This contradicts the assumption that $b \in \langle a \rangle$ and $\langle bab^{-1}\rangle = \langle a \rangle$. Hence every subgroup of the form $\langle a \rangle$ is normal in $G$. Hence satifying the condition of the theorem. $G = \langle x \rangle \oplus \langle b \rangle \cong \mathbb{Z}_p \oplus \mathbb{Z}_{p}$
    
\end{enumerate}
\end{document}